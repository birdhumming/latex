\documentclass[12pt]{exam}
\usepackage{amsmath,amsfonts,amssymb}

\usepackage{diagbox}
\usepackage[colorlinks]{hyperref}

\usepackage{graphicx,caption,subcaption}

\usepackage{tikz}

%\usepackage{url}
%\urlstyle{tt}

\usepackage{geometry}
\geometry{%
  letterpaper,
  lmargin=1.5cm,
  rmargin=1.5cm,
  tmargin=2cm,
  bmargin=2cm,
  footskip=12pt,
  headheight=12pt}
 
 
\usepackage[parfill]{parskip}
\newcommand{\ATAN}{\operatorname{TAN}^{-1}}
\DeclareMathOperator{\COS}{COS}
 
\usepackage{lastpage}
\headheight 35pt

\lhead{Mathcounts 2020}

\def\a{{\alpha}}
\def\b{{\beta}}
\def\g{{\gamma}}
                   
\DeclareUnicodeCharacter{2212}{-}  

\qformat{}

\begin{document}

\thispagestyle{empty}

\begin{center}\section*{AMC10 2019 A4}

\end{center}
\bigskip
\begin{questions}

\question 
A box contains $28$ red balls, $20$ green balls, $19$ yellow balls, $13$ blue balls, $11$ white balls, and $9$ black balls. 
What is the minimum number of balls that must be drawn from the box 
without replacement to guarantee that at least $15$ balls of a single color will be drawn?
\bigskip

\begin{oneparchoices}
    \choice 75
    \choice 76
    \choice 79
    \choice 84
    \choice 91
\end{oneparchoices}

\vspace{0.5cm}
The maximum number of balls we can draw without drawing 15 of the same color 
comes from this scenario: 
we draw 14 red, 14 green, 14 yellow, 13 blue, 11 white, and 9 black. 
This gives us 75 balls. Drawing one more ball will guarantee us 15 of the same color, 
so the answer is B (76).

\end{questions}
\end{document}