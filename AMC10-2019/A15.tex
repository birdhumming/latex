\documentclass[12pt]{exam}
\usepackage{amsmath,amsfonts,amssymb}
\usepackage{diagbox}
\usepackage[colorlinks]{hyperref}
\usepackage{graphicx,caption,subcaption}
\usepackage{tikz}
\usepackage{geometry}
\geometry{%
  letterpaper,
  lmargin=1.5cm,
  rmargin=1.5cm,
  tmargin=2cm,
  bmargin=2cm,
  footskip=12pt,
  headheight=12pt
}
\usepackage[parfill]{parskip}
\newcommand{\ATAN}{\operatorname{TAN}^{-1}}
\DeclareMathOperator{\COS}{COS}
\usepackage{lastpage}
\headheight 35pt

\lhead{Mathcounts 2020}

%\def\a{{\alpha}}
%\def\b{{\beta}}
%\def\g{{\gamma}}
                   
\DeclareUnicodeCharacter{2212}{-}  

\qformat{}

\begin{document}

\thispagestyle{empty}

\begin{center}\section*{AMC10 2019 A15}

\end{center}
\bigskip
\begin{questions}

\question 
A sequence of numbers is defined recursively by $a_1 = 1$, $a_2 = \frac{3}{7}$, and 
\[a_n=\frac{a_{n-2} \cdot a_{n-1}}{2a_{n-2} - a_{n-1}}\]for all $n \geq 3$.
Then, $a_{2019}$ can be written as $\frac{p}{q}$, where $p$ and $q$ are relatively prime positive integers. 
What is $p + q$?
\bigskip

\begin{oneparchoices}
    \choice 2020
    \choice 4039
    \choice 6057
    \choice 6061
    \choice 8078
\end{oneparchoices}

\vspace{0.5cm}
Using the recursive formula, we find $a_3 = \frac{3}{11}$,
$a_4 = \frac{3}{15}$, and so on. It appears that $a_n = \frac{3}{4n - 1}$
for all $n$. Setting $n = 2019$, we find $a_{2019} = \frac{3}{8075}$, so the 
answer is E (8078).

\vspace{0.5cm}
By our assumption, $a_{m-1} = \frac{3}{4m-5}$ and $a_m = \frac{3}{4m-1}$. 
\[a_{m+1} = \frac{a_{m-1} \cdot a_m}{2a_{m-1} - a_m} 
= \frac{\frac{3}{4m-5} \cdot \frac{3}{4m-1}}{2 \cdot \frac{3}{4m-5} - \frac{3}{4m-1}} 
= \frac{\frac{9}{(4m-5)(4m-1)}}{\frac{6(4m-1)-3(4m-5)}{(4m-5)(4m-1)}} 
\] 

\[ = \frac{9}{6(4m-1) - 3(4m-5)} 
= \frac{3}{4(m+1) - 1}\] so our induction is complete.


\end{questions}
\end{document}