\documentclass[12pt]{exam}
\usepackage{amsmath,amsfonts,amssymb}
\usepackage{diagbox}
\usepackage[colorlinks]{hyperref}
\usepackage{graphicx,caption,subcaption}
\usepackage{tikz}
\usepackage{geometry}
\usepackage{textgreek}
\usepackage[T1]{fontenc}
\geometry{%
  letterpaper,
  lmargin=1.5cm,
  rmargin=1.5cm,
  tmargin=2cm,
  bmargin=2cm,
  footskip=12pt,
  headheight=12pt
}
\usepackage[parfill]{parskip}
\newcommand{\ATAN}{\operatorname{TAN}^{-1}}
\DeclareMathOperator{\COS}{COS}
\usepackage{lastpage}
\headheight 35pt

\lhead{Mathcounts 2020}

\def\a{{\alpha}}
\def\b{{\beta}}
\def\g{{\gamma}}
                   
\DeclareUnicodeCharacter{2212}{-}  

\qformat{}

\begin{document}

\thispagestyle{empty}

\begin{center}\section*{AMC10 2019 A4}

\end{center}
\bigskip
\begin{questions}

\question 
Melanie computes the mean $\mu$, the median M, and the modes of the 365 values 
that are the dates in the months of 2019. 
Thus her data consists of 12 1s, 12 2s, ... , 12 28s, 11 29s, 11 30s, and 7 31s. 
Let d be the median of the modes. 
Which of the following statements is true?
\bigskip

\begin{oneparchoices}
    \choice $\mu < d < M$
    \choice M < d < $\mu$
    \choice d = M = $\mu$
    \choice d $<$ M < $\mu$
    \choice d < $\mu$ < M
\end{oneparchoices}

\vspace{0.5cm}
d obviously must be less than M, 
so we can rule out (B) and (C). 
Since there are 365 entries, M is the 183rd number, 
which lies between 15$\times$12 + 1 (180) and 16$\times$12 (192). 
Therefore, M=16. We can also see that d = 14.5. 
If we ignored the 11 29s, 11 30s, and 7 31s, 
$\mu$ would also be =14.5. Because there are larger elements given, 
though, we can see that $\mu$ is greater than d. 
On the other hand, since there are fewer 29s, 30s, and 31s than the rest of the numbers,
 the mean has to be lower than the median. 
 So, the answer is E (d < $\mu$ < M).

\end{questions}
\end{document}